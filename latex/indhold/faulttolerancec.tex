\section{Fault Tolerance}
A large web application usually utilize multiple servers in order to fulfill its designed purpose and this fact can lead to errors caused by the communication between the servers and services. These errors can cause a lot of problems if left unattended. The solution to this problem could be to recover from them instead of trying to prevent them. To recover from this communication problems it is essential that the communication it done autonomously so the data it kept consistent. Transactions is used to obtain this property, which is known as a term in the world of databases.

So in order to decrease the fault rate and keep our data consistent in the web application we have to implement recovery. This is a consequence of having communication between multiple servers and services. As earlier stated we utilizes the REST principle between the back-end and front-end which means that communication between this two ends already implements a form of autonomously in terms of the call to the back-end, but this does not take other communications into consideration. An example could the communication between the back-end and the database or the front-end and a server which provides the web application with some sort of relevant content. To implement recovery on these communication channels we have to choose the protocol we want to use. We selected the 2phases protocol because it provides the necessary recovery and it fits our communication setting, our application talks with less of 5 services at the time. Another solution would be to connect to the other servers directly through hardware, but because of location this can become highly impractical.

For 2phased protocol there are different optimization strategies that solves some of the problems with 2phased protocol. The consequences of the choice of protocol would require additional logging because the system most have some sort of way to recreate its state after a crash.