\section{SPARQL}
SPARQL is language used to query search results from a given RDF data set. It works by selecting some data attributes from the data set. These result are restricted by the "where" closure, where the user can specify one or more predicates to find the result which they want. It is also possible to restrict the result in term of size by the "filter" option.

We used SPARQL to search through different data sets to see how it actual works and the results would be from the different searches. We think that SPARQL provides good qualities in terms of searching through large RDF data sets by allowing us to design very specific queries. The general flexibility in terms of end result is also a nice feature that adds the possibilities to customize the results to fit once need.

But as a side note we also think that it can a bit confusing setting the query up for getting the right result.