\section{Implementation of AB-testing}
In AB-testing, users are randomly exposed to two versions of a page, giving the possibility to compare the results, finding the best version of the site. This structure can be seen of \Cref{fig:ABtesting}.

\figur{1}{ABTesting.png}{Figure illustrating the structure of AB-testing, taken from the course slides from lecture 8.}{fig:ABtesting}

If we were to implement AB-testing in our project, we would need resources which are not available for this mini project. It could be possible for us to implement the coding part of AB-testing if given enough time, but we do not have the users for our page, compared to the number needed for performing an AB-test. 

Assuming we had resources, AB-testing could be implemented by making two different views of a page, e.g. the edit country page. This could easily be done, as it is only the view that should be changed, and the functionality could be preserved by the model and controller. We could then make two views of the edit country page, with two different layouts, and measure the speed, register the user behavior when e.g. they are changing the country or if different pages of the application are reachable etc. These criteria could be our overall evaluation criteria (OEC) and can be used to improve the page by on the testing results. To preform the AB-test it would require us to setup a hypothesis which could be e.g. That the treatment (new view) reduces the time spent by a user on a given page by \textit{X} amount. Will also have to think about the confidence and power of our tests.

After setting up the test or experiment, the users have to be distributed evenly without any bias towards any of the sides. We would choose to use Server-side-assignment because it would not require any additional hardware, it has little impact of the users experience and later experiences can be conducted at a low cost. Downside to this choice it has a high implementation cost to begin with but we think the ability to conduct subsequent experience at low cost is worth it.