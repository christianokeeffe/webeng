\section{Scalability Strategy}
A web store or other kind of web applications income in someway origins from the users and therefore it is important to provide a nice user experience. Therefore it is important that the users can easily access the web application without high latency or unavailable pages due to high traffic on the server. Therefore it is important that the web application is scalable so it possible to scale the application to more servers in order to handle more traffic and thereby more customers.

General when scaling an application it is a requirement to redirect the users, to the servers which are not stressed, before it makes any sense. For this there exists different techniques e.g. redirection, round robin DNS, load balancing switch and load balancing proxy. With this the hardware is able to handle the extra traffic but the software still needs to handle the scaling too.

A method for scaling an application is to use replication of the database, so the database and the application code is located on each of the servers. This increases the traffic that the web application can handle by allowing more accesses to the data and more processing power. A down side to this implementation is that you need to handle the synchronization between the databases in order for theirs data to be consistent. Another downside is a lot of duplicated data is stored across multiple databases.

Another method for scaling is called content aware and centers around have a set of sub databases containing the data for each of the applications pages and a central database containing all the data. This approach allows the application to scale and rises the user capacity while keeping the data redundancy low. A downside is that a server still can experience heavy load if a lot of users access the same specific page. A solution to this problem could be to utilize database replication of the pages with a lot of daily access but it would require additional work.

A third method could be to utilize catching to minimize the number of queries to the central database. The idea is to try and catch as much of relevant information to the users activity to minimize the need for new queries. The upside to this approach is that it allows for scaling without the need for storing duplicated data and low latency at the front-end. A downside to this method is that the catching needs to implemented in a smart way so it does not consumes all the user memory or it stores to little to actually work.

We choose to use catching because the structure of application fits the general concept of catching in that if e.g. the user looks for different cities in the city table, where it ideal to store all the cities at the user. With this choice we also evade the issue of have to deploy multiple databases needed for the other methods.