A very common design issue of an application, is how to navigate through and present data.  We have in our application chosen a navigation design, as well as en presentation for our different views.

\section{Navigation Designs}
There are a number of different design for navigation.

\begin{itemize}
\item \textbf{Step Navigation}: Is navigation which works in steps, like arrows beneath a picture in a slide show or gallery.
\item \textbf{Paging Navigation}: A number of pages to step through, like a search result.
\item \textbf{Breadcrumb Trail}: A list of the parent categories of navigation.
\item \textbf{Tree Navigation}: A tree structure, where you can explore the different subtrees in the navigation.
\item \textbf{Site Maps}: A map of the sites content, set up like a folder structure.
\item \textbf{Directories}: A list of directories.
\item \textbf{Tag Clouds}: A number of tags, which can vary in size according to their use.
\item \textbf{A-Z index}: An alphabetic navigation.
\item \textbf{Navigation Bars and Menus}: A top bar, with links to navigate.
\item \textbf{Vertical Menu}: A vertical menu, often placed in the side, with links to navigate.
\item \textbf{Dynamic Menu}: A possibility to dynamically fold our a new sub menu when hovering over an item.
\item \textbf{Drop Down Menu}: A drop down menu with the possibility to select an item.
\end{itemize}

These are some of the different navigation designs to chose from when designing your navigation.

\section{Navigation Design Techniques}
There are different design techniques to represent the selected design.
We will describe and compare the "Navigation Views" and "WebML Hypertext Schema".

\begin{itemize}
\item \textbf{Navigation Views}: Navigation views consists of a list of items, which makes then suitable for querying. They are defined as object orientated views, with view name, base data, and other data.
\item \textbf{WebML Hypertext Schema}: WebML shares the notation of the elements with the navigation views, but it differs as it enables different ways for selecting data within navigation nodes.
\end{itemize}

WebML can be used to more complex applications than the simple navigation views, and have many possibilities for extension compared to the navigation views. The navigation views is more simple and is suitable for minor or simple applications compared to WebML.