\section{Preventing Attacks in our Web Application}
As we have determined in previous section, our system contains security breaches. To fix a lot of the security breaches, we could follow the defensive practice and evaluate the input of all the input fields and check to see if their content can not be parsed, if they contain characters to escape a string or if they contain any SQL statements. A flaw to this approach is it require us, the programmers, to actively remember to preform the checks at each input. Another practices we could use is the detection and prevention practice were you analyze or find patterns in the SQL request and only execute the request that does not raise an alert.

There are no login on our web application which means everybody can access functionality in the web application to change the database which could be exploited. Therefor by implement login and sessions we can make it harder for the attackers to gain access to database. By implementing sessions we also improve the security of the API, by validating the API calls and check if they are valid. 