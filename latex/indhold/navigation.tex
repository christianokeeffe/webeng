\section{Navigation}
We chose to use navigation view design to create an abstract overview of have our web application. This should provide us with an idea and a way to describe how the navigation would work in our web application.

We start by creating the navigation views for the different pages on our web application. We will use one concrete example to describe the general process. The chosen example is the country view of our application, which is responsible for showing all the countries from the database. This view contains a list of \textit{country} which have \textit{Name}, \textit{ID} and \textit{Population} as attributes. These attributes should be displayed in the tables columns. Each \textit{country} then links into another view were it is possible to edit the given \textit{country}. The country view also contains a link to a add view were it is possible to add countries to the database.

The navigation context scheme is used to describe the context of the navigation views.

We have chosen to use a navigation menu at the top of our web application so the users have fast access to the different features that our web application provides.

We uses vertical menus to provide an overview of available functions on a given page and a navigation menu to navigate through the different pages.