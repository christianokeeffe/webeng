\section{Choice of Data Set and its Structure}
We have decided to use the Mondial data set structure as a base for the database structure. We choose this geographical data set over the others because we found it well structured and suitable to implement as a database. The Mondial data is structured in a hierarchy which relative ease can be projected into a database and insures a relative low data redundancy. A issue that could arise would be connected to the low redundancy which can lead to more joins to achieve the desired information. We choose the low redundancy approach to lower the space usage and also the observation that relevant data would not involve a lot of join actions.

The Modial data sets hierarchy starts with declaration of continents. Then each country refers to the continent it resides in. Each country then have a series at city connected to it or provinces which contains the cities. Each level in the hierarchy have a set of attributes that describes information.